\chapter{\uppercase{Conclusió}}
Per desenvolupar el projecte s'han consultat fonts fiables en el camp de la medicina. Tot i ser una disciplina de la qual tenim idees molt vagues, hem entès com són les diferències de tensió que es poden captar en diferents llocs del cos humà. Un cop parlem de tensions, parlem d'electrònica, camp en què sí que tenim coneixements.\\
\newline El consum d'energia i les dimensions del hardware utilitzat han estat dues "constraints" que hem tingut en compte al llarg del desenvolupament. S'ha optat per una pila d'alta densitat energètica per tal d'optimitzar la relació energia/volum. Tot i utilitzar plaques de desenvolupament, s'ha aconseguit un volum de pocs centímetres cúbics, menor del de molts electrocardiògrafs comercials.\\
\newline La programació dels microcontroladors s'ha fet per tal de fer en paral·lel les tasques d'adquirir dades i la de mostrar-les al client. D'aquesta manera augmenta la robustesa del sistema. S'ha programat una web en HTML que inclou estils definits en CSS i contingut dinàmic en JavaScript. Considerem que té una interfície atractiva i intuïtiva.\\
\newline El client podrà conèixer amb precisió la freqüència cardíaca del pacient o persona que porti aquest aparell mitjançant una pàgina web. A la web apareixen les últimes dades de freqüència cardíaca així com les de l'últim dia. L'anell de LEDs és una forma molt visual de saber quina freqüència cardíaca té la persona.\\
\newline Els autors considerem que s'han complert satisfactòriament els objectius marcats a l'inici del projecte.

\vspace*{\fill}
\noindent Llorenç Fanals Batllori\\
Pol Fernández Rejón\\
Graduats en Enginyeria Electrònica Industrial i Automàtica\\
\\
\\
Girona, 14 de gener de 2020.

\clearpage